\documentclass[11pt,a4paper]{moderncv}
\moderncvtheme[black]{classic}
\usepackage[scale=0.8, top=1cm, bottom=1cm, left=1.5cm, right=1.5cm]{geometry}
\usepackage[utf8]{inputenc}
\usepackage[french]{babel}

\usepackage{graphicx}
\usepackage[scaled]{helvet}
% Help for calculating age
\usepackage{xifthen}

\usepackage[T1]{fontenc}
\usepackage{lmodern}
\usepackage{color}
\usepackage{hyperref}

% Only if the base font of the document is to be sans serif
\renewcommand*\familydefault{\sfdefault}
% Don't print page numbers
\nopagenumbers{}
% Title columns
\setlength{\hintscolumnwidth}{3cm}
\AtBeginDocument{\recomputelengths}

\definecolor{roseSG}{RGB}{185,61,135}
\providecommand{\smartandgeek}{Smart\textcolor{roseSG}{\&}Geek}

% Age calculation
\newcounter{dobyear}\setcounter{dobyear}{1989}
\newcounter{dobmonth}\setcounter{dobmonth}{12}
\newcounter{dobday}\setcounter{dobday}{19}
\newcounter{age}\setcounter{age}{\the\year - \thedobyear}
%this compares your’s birth year to the current year and stores the value in a new counter
\newtest{\conditiona}{%
\cnttest{\the\month}{=}{\thedobmonth} \AND %
\cnttest{\the\day}{>=}{\thedobday}%
}
%this sets the condition “if the current month is the same as your birth month and the current day is greater or equal than your birth day
\newtest{\conditionb}{%
\cnttest{\the\month}{>}{\thedobmonth}%
}
%this sets the condition: “if we the current month is after your birth month“
\ifthenelse{\conditiona \OR \conditionb}{}{\addtocounter{age}{-1}}
%this adjusts your age if one of the two conditions above is true, i.e. current time is later than your birthday of this year.
% /Age calculation

\renewcommand*{\quotefont}{\small\slshape}

\firstname{Florian}
\familyname{Levis}
\title{Développeur, avec un l et deux p}
\address{}{95170 Deuil-la-Barre, France}
\phone[mobile]{+33 6 84 94 25 85}
\email{levis.florian@gmail.com}
\homepage{levisflorian.name}
\social[github]{Gounlaf}
\social[linkedin]{gounlaf}

\extrainfo{\theage{} ans, permis B, véhiculé\\mais en métro ou à pieds c'est mieux}
%\photo[64pt]{photo.jpg}
\photo{photo.jpg}
\quote{$\pi\ \%\ 1337\ =\ 42$}

\begin{document}

\makecvtitle

\section{Expériences professionnelles}

\cventry{2018--Maintenant}{Développeur back-end}{\href{https://alphanetworks.tv/}{Alphanetworks}}{Paris 75001}{CDI}{
    \begin{itemize}
        \item Responsabilités :
        \begin{itemize}
            \item encadrement d'une apprentie
            \item encadrement d'une junior (reconversion professionnelle)
            \item lead technique sur \emph{Accurate EPG} et \emph{ENR}
        \end{itemize}
        \item Accurate EPG - Plateforme de correction EPG en live, utilisée principalement par des opérateurs télécoms :
        \begin{itemize}
            \item backend en microservices :
            \begin{itemize}
                \item APIs internes en REST et en gRPC (Kotlin, PHP, SQL, RabbitMQ, Gradle)
                \item processing live de +/- 40 chaînes de télévision (FFmpeg, C\#, Python/ML, RabbitMQ)
                \item tests unitaires et tests d'intégration (phpunit, junit, mockito)
                \item metrics pour le suivi en production (prometheus, micrometer, Grafana)
                \item conteneurisation et déploiements via charts helm (Docker/Kubernetes)
            \end{itemize}
            \item maintenance du front-end utilisé en continu (365j/365, 7j/7, 24h/24) par les équipes de production (Reactjs, conteneurisation)
            \item développement de \emph{Proof of Concepts}
            \item développement d'outils internes pour la R\&D (timelines EPG, visualisation de \emph{fingerprints})
            \item mise en place et maintenance de l'intégration continue (Gitlab CI)
        \end{itemize}
        \item ENR - Enrichissement vidéo :
        \begin{itemize}
            \item détection intro/outro (Kotlin, FFmpeg, C\#)
            \item sélection automatique de scènes pour illustrer une vidéo (Kotlin, Python, SQL)
            \item développement de \emph{Proof of Concepts}
            \item développement de \emph{librairies} internes (publication sur le repository du Gitlab)
            \item mise en place et maintenance de l'intégration continue (Gitlab CI)
            \item conteneurisation (Docker/Kubernetes)
        \end{itemize}
        \item \href{https://www.alphanetworks.tv/en/products/tucano/overview}{Tucano - Le logiciel vidéo modulaire} (Java / Spring Boot 3, SQL, Docker/Kubernetes) :
        \begin{itemize}
              \item développement de nouvelles fonctionnalités (intégration de Stripe Checkout)
              \item correction de bugs
        \end{itemize}
    \end{itemize}
}

\cventry{2022--2023}{Maître d'apprentissage}{\href{https://alphanetworks.tv/}{Alphanetworks}}{Paris 75001}{}{Maître d'apprentissage de \href{https://www.linkedin.com/in/linda-sadaoui/}{Linda S.} sur la formation "Développeur Web" d'OpenClassrooms}

\cventry{2021--2023}{Mentor}{\href{https://openclassrooms.com/}{OpenClassrooms}}{Distanciel}{}{Mentorat d'étudiants sur la formation "Développeur PHP/Symfony" d'OpenClassrooms}

\cventry{2019--2023}{Jury Professionnel}{DIRECCTE / Ministère du Travail}{}{}{Jury professionnel pour le Titre Professionnel Développeur Web \& Web Mobile (\mbox{TP-01280m03})}

\cventry{2016--2018}{Développeur Android}{Mobile Media Com}{Paris 75001}{CDI}{Développement d'applications de \emph{streaming vidéo} (similaires à YouTube) sous Android compatibles téléphone, tablettes et télé (Android TV) : \href{https://www.playzer.fr}{Playzer}, Replay TV France
    \begin{itemize}
        \item première expérience professionnelle de développement sous Android
        \item ré-écriture en Kotlin d'applications existantes en Java
        \item intégration de l'ExoPlayer (via \href{https://github.com/brianwernick/ExoMedia}{ExoMedia})
        \item intégration de solutions de paiement opérateur (sur BBox Miami)
        \item intégration des designs
    \end{itemize}
}

\cventry{2011--2016}{Développeur fullstack}{\smartandgeek}{Paris 75020}{Contrat d'apprentissage en Alternance (1 an) puis CDI}{
    \begin{itemize}
	    \item Développement d'applications web et mobiles, dont :
        \begin{itemize}
            \item projet JobAroundMe (PHP/Zend Framework 2)
            \begin{itemize}
                \item première expérience de \emph{lead developer} : équipe de 5 personnes auto-gérée en Roumanie
                \item conception technique de la version web et des applications mobiles (Android/iOS),
            \end{itemize}
            \item projet Dynamic-Cover (PHP/Zend Framework 2, Fabric.js)
            \begin{itemize}
                \item service de création de \textit{covers} paramétrables sur pages Facebook (évènements à une date fixe, compte à rebours, nombre d'abonnés\ldots)
            \end{itemize}
            \item nombreux jeux concours pour pages Facebook de grandes marques (PHP/Framework maison, Vanilla JS)
        \end{itemize}
        \item Administration système :
        \begin{itemize}
            \item administration des serveurs (staging/production sur AWS, Ubuntu Server)
            \item administration du système de versioning (Gitlab - self-hosted)
        \end{itemize}
    \end{itemize}
}

\cventry{2010--2011}{Développeur back-end/front-end}{3DS (Dassault Systèmes)}{Vélizy-Villacoublay 78140}{Contrat d'apprentissage en Alternance}{
    Gestion du parc informatique (ecosystème Microsoft : C\#, SQL Server)
    \begin{itemize}
        \item nouvelle fonctionnalité notable : génération de PDFs à la volée pour la gestion des changement de matériel
        \item correction de bugs
    \end{itemize}
}

\cventry{2008--2010}{Développeur back-end/front-end}{\href{https://www.maecia.com}{Maecia}}{Paris 75002}{Contrat d'apprentissage en Alternance}{Développement de sites e-commerce et de communications (PHP (eZ Publish, Joomla!), SQL (MySQL/MariaDB), JavaScript (Mootools)}

\section{Formations}

\cventry{$\infty$}{Auto-formation, veille technologique}{}{}{}{}

\cventry{2023}{Formation \emph{manager}}{}{}{}{Formation d'une journée dans le cadre de mon status de maître d'apprentissage.}

\cventry{2010--2013}{Manager en Ingénierie Informatique (BAC+5)}{ITIN}{CCIV}{}{Options : Génie des Logiciels Embarqués (M1) / Jeux-vidéos (M2)}

\cventry{2008--2010}{BTS Informatique de Gestion (BAC+2)}{ITIN}{CCIV}{}{Option : Développeur d'applications}

\cventry{2007--2008}{1ère année Licence Science du Vivant}{Université Paris 7}{}{}{Objectif avorté : spécilisation en bioinformatique}

\cventry{2007}{Baccalauréat Scientifique}{Lycée Charles Baudelaire}{95470 Fosses}{}{}

\section{Compétences informatiques \& Langages de programmation}

\cvitem{\textbf{Backend}}{Kotlin (Spring Boot 3, natif/cinterop, Android), Java (Spring Boot 3), gRPC, PHP (5-8 ; Symfony 3/4/5, Laravel), C\#, JavaScript/TypeScript (Node.js), C/C++}

\cvitem{\textbf{Frontend}}{TypeScript, JavaScript (Reactjs, jQuery), HTML5 \& CSS3}

\cvitem{\textbf{Base de données}}{PostgreSQL/MySQL/MariaDB (+ liquibase), SQL Server, SQLite (Android), Oracle (scolaire)}

\cvitem{\textbf{Conteneurisation}}{Docker, Kubernetes (création et utilisations de \emph{charts helm})}

\cvitem{\textbf{Opensource}}{\href{https://github.com/dlemstra/Magick.NET}{Magick.NET}, \href{https://github.com/Gounlaf/Magick.KT}{Magick.KT} (portage sur Kotlin de Magick.NET via cinterop)}

\cvitem{\textbf{Embarqué}}{Traitement audio sur ARM Cortex-M4 (Freescale)}

\cvitem{\textbf{Divers}}{\mbox{Ubuntu}, \mbox{JetBrains All Products Pack}, \mbox{macOS}, \LaTeX{} (mon CV n'est pas moche !)}

\section{Langues / Informations complémentaires}

\cvitemwithcomment{Anglais}{Niveau BAC+2}{TOEIC 2011 : 810 points ; anglais technique lu/écrit quotidiennement}

\cvitem{Loisirs}{Pour la santé physique je pratique l'escalade (passport blanc), un sport qui demande de la patience, de la persévérence, et de la confiance ; j'ai aussi pratiqué la plongée sous-marine (Niveau 2 FFESSM) pendant ma jeunesse, un sport qui permet de s'évader ; pour l'intellect, je joue à divers jeux vidéos}

\end{document}
